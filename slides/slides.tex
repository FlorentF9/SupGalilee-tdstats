\documentclass{beamer}
\usepackage[T1]{fontenc}
\usepackage{url}
\usepackage{hyperref}
\usepackage{xcolor}
\usepackage[french]{babel}
\usepackage{appendixnumberbeamer}
\usepackage{ulem}
\usepackage[os=win]{menukeys}
\usetheme[sectionpage=none,
          subsectionpage=progressbar,
          numbering=fraction,
          progressbar=none,
          background=light]{metropolis}

\title{Introduction à Python pour la Data Science}
\subtitle{Cours de statistiques descriptives --- TD}
\date{2020}
\author{\textsc{Florent Forest}\vspace{0.2cm}\\
\includegraphics[height=0.35cm]{./rc/e-mail-envelope-blue}\;\scriptsize{\href{mailto:forest@lipn.univ-paris13.fr}{forest@lipn.univ-paris13.fr}}\\
\includegraphics[height=0.35cm]{./rc/grid-world-blue}\;\scriptsize{\href{http://florentfo.rest}{http://florentfo.rest}}\\
\includegraphics[height=0.35cm]{./rc/github-logo-blue}\;\scriptsize{\href{https://github.com/FlorentF9}{FlorentF9}}\\
}
\institute{\vfill\hfill
\includegraphics[height=1.0cm]{./rc/up13_logo}
\hspace{0.1cm}
\includegraphics[height=1.0cm]{./rc/supgalilee_logo}}

\begin{document}

  \maketitle

  \begin{frame}{Présentation du cours}

    \textbf{Partie pratique du cours de Statistiques Descriptives (MACS 3).}

    \metroset{block=fill}
    \begin{block}{Objectifs}
      \begin{itemize}
        \item[>] Se familiariser avec le langage Python et les notebooks Jupyter
        \item[>] Maîtriser les bases des modules de calcul numérique, d'analyse de données, de visualisation et d'apprentissage (\texttt{numpy}, \texttt{pandas}, \texttt{matplotlib}, \texttt{scikit-learn})
        \item[>] Mettre en pratique des méthodes statistiques 
        en réalisant des cas d'études concrets issus du domaine aéronautique
      \end{itemize} 
    \end{block}

  \end{frame}

  \begin{frame}{Organisation}

    \begin{itemize}
      \item[>] 6 séances de TD (jeudi 14h-17h \small{\textcolor{red}{sauf jeudi 24/09, décalé à vendredi 25/09}})
      \item[>] \sout{En salle info} \textcolor{red}{À distance avec séances correction/questions individuelles en visio, sur inscription (Framadate)}
      \item[>] TDs corrigés publiés au fur et à mesure (Github)
    \end{itemize}

    \metroset{block=fill}
    \begin{block}{Les outils}
      \centering
      \includegraphics[height=1cm]{./rc/jupyter.png}\;
      \includegraphics[height=1cm]{./rc/colab.png}\\
      Aucune installation requise. Seul pré-requis : connexion internet et \textbf{compte Google} (pour synchro Drive).
    \end{block}

    $\blacktriangleright$ \textbf{\href{https://github.com/FlorentF9/SupGalilee-tdstats}{github.com/FlorentF9/SupGalilee-tdstats}}\\
    $\blacktriangleright$ \textbf{\href{https://colab.research.google.com}{colab.research.google.com}}\\
    $\blacktriangleright$ \textbf{\href{https://framadate.org/SupGalilee-tdstats}{framadate.org/SupGalilee-tdstats}}
    
  \end{frame}

  \begin{frame}{Utilisation du Framadate}

    \begin{center}
      $\blacktriangleright$ \textbf{\href{https://framadate.org/SupGalilee-tdstats}{framadate.org/SupGalilee-tdstats}}
    \end{center}

    \begin{itemize}
      \item Mot de passe : \textcolor{red}{\texttt{*******}}
      \item Un TD de 3h est divisé en 12 créneaux de 15 min. S'inscrire sur un créneau, \textcolor{red}{si possible à chaque séance} et \textcolor{red}{en avance}.
      % \item \textcolor{red}{Pas d'inscription = absence} (sauf justifié par e-mail).
    \end{itemize}

  \end{frame}

  \begin{frame}{Utilisation de Colab}

    \begin{center}
      \includegraphics[height=1cm]{./rc/colab.png}\\
      $\blacktriangleright$ \textbf{\href{https://colab.research.google.com}{colab.research.google.com}}
    \end{center}

    \begin{enumerate}
      \item Cliquez sur l'onglet \menu{GitHub}
      \item Dans la barre de recherche, tapez \texttt{FlorentF9}, puis sélectionnez le dépôt \texttt{SupGalilee-tdstats}.
      \item Cliquez sur un TD pour l'ouvrir (\texttt{TDX-eleve.ipynb})
      \item Vérifiez que vous êtes connecté à votre compte Google (icône en haut à droite).
      \item Cliquez sur \menu{Copier sur Drive} afin de pouvoir sauvegarder vos modifications.
      \item \menu{Partager} avec \texttt{florent.forest9@gmail.com} pour collaborer en temps réel.
    \end{enumerate}

  \end{frame}

  \begin{frame}{Utilisation de Colab (suite)}

    Un notebook est constitué de \emph{cellules} pouvant contenir du texte (Markdown/TeX/HTML) ou du code (ici Python).

    \begin{block}{Commandes de base et raccourcis}
      \small{
      \begin{tabular}{ll}
        Clic ou \keys{\arrowkeyup}/\keys{\arrowkeydown} & Sélectionner et changer de cellule.\\
        Clic ou \keys{\return} & Entrer en mode \emph{modification}. \\
        \keys{\esc} & Sortir du mode \emph{modification}. \\
        \keys{\shift+\return} & Exécuter une cellule. \\
        \keys{A}/\keys{B} & Insérer une cellule de code au-dessus/en-dessous.\\
        Clic sur \menu{+ Texte} & Insérer une cellule de texte.\\
        \keys{\ctrl+M} \keys{D} & Supprimer une cellule.\\
        \keys{\ctrl+M} \keys{J}/\keys{K} & Déplacer une cellule vers le bas/haut.\\
      \end{tabular}
      }
    \end{block}

  \end{frame}

  \begin{frame}{Planning prévisionnel}
    \centering
    \begin{enumerate}
      \item[17/09] Présentation
      \item[18/09] TD1
      \item[24/09] correction TD1 + TD2
      \item[01/10] correction TD2 + TD3
      \item[08/10] TD3
      \item[15/10] correction TD3 + TD4
      \item[22/10] correction TD4 + discussions finales 
    \end{enumerate}
  \end{frame}

\end{document}
